% imports
\usepackage{texor/texor}
\usepackage{hyperref}
\usepackage{marginnote}

\makeatletter

%%% PRELUDE %%%

\newcommand{\infty@abort}{
  \PackageError{18.90infty}
}

%%% TAGS %%%

% create a new tag
% args: tag name, tag type, tag number
\newcommand{\infty@tag@new}[3]{
  % check if this tag has been defined already,and error if it has
  \@ifundefined{infty@tag@canary@#1}{}{
    \infty@abort{tag '#3' defined already}{}
  }
  
  % check if this tag type is valid
  \@ifundefined{infty@tag@symbol@#2}{
    \infty@abort{tag type '#2' not found}{}
  }{}

  \expandafter\gdef\csname infty@tag@canary@#3\endcsname{#1}
  \expandafter\gdef\csname infty@tag@type@#1\endcsname{#2}
  \expandafter\gdef\csname infty@tag@num@#1\endcsname{#3}
}

% create a new tag type
% args: tag type name, tag type symbol
\newcommand{\infty@tag@newtype}[2]{
  \expandafter\gdef\csname infty@tag@symbol@#1\endcsname{#2}
}

\newcommand{\infty@tag@symoftype}[1]{\texor@call{infty@tag@symbol@#1}}

\newcommand{\infty@tag@type}[1]{\texor@call{infty@tag@type@#1}}
\newcommand{\infty@tag@num}[1]{\texor@call{infty@tag@num@#1}}
\newcommand{\infty@tag@sym}[1]{\infty@tag@symoftype{\infty@tag@type{#1}}}
\newcommand{\infty@tag@margin}[2][]{\marginnote{\tt \normalsize <\infty@tag@num{#2}>}}

\newcommand{\tagsymbol}[1]{%
  \infty@tag@labeled{\infty@tag@num{#1}}{\infty@tag@sym{#1}}%
}

% reference a tagged item, e.g.
% \tagref{thm:thom-isomorphism}
\newcommand{\tagref}[1]{%
  \@ifundefined{infty@tag@canary@#1}{%
    {\tt <ref:#1>}%
  }{%
    \hyperref[#1]{%
     \tagsymbol{#1} \ref*{#1}%
    }%
  }%
}

\newcommand{\infty@tag@labeled}[2]{%
  \def\infty@tag@label{{\tiny {\tt <#1>}}}%
  $\mathrlap{\expandafter\raisebox{-4pt}{\parbox[t][0pt]{\widthof{\infty@tag@label}}{\infty@tag@label}}}{\text{#2}}$%
}

%%% SECTIONS %%%

% wrap a section macro to include a tag
% args: name macro pretty-name
\newcommand{\infty@sec@wrap}[3]{
  % define a new tag type
  \infty@tag@newtype{#1}{#3}

  \expandafter\let\csname infty@sec@wrap@temp@#1\endcsname #2

  % args: label tag pretty-name
  \renewcommand{#2}[3]{%
    % create a new tag
    \infty@tag@new{#1:##1}{#1}{##2}%

    % call the wrapped macro
    \texor@call{infty@sec@wrap@temp@#1}{##3\infty@tag@margin{#1:##1}}

    % add the label
    \label{#1:##1}%
  }
}

\infty@sec@wrap{cha}{\chapter}{Chapter}
\infty@sec@wrap{sec}{\section}{Section}

%%% THEOREMS %%%

\newcommand{\infty@thm@new}[2]{
  % clear anything set by amsthm or texor
  \expandafter\undef\csname #1\endcsname
  \expandafter\undef\csname end#1\endcsname
  \expandafter\undef\csname c@#1\endcsname  

  % definie a new tag type
  \infty@tag@newtype{#1}{#2}

  % invoke the inner amsthm bit
  \newtheorem{infty@thm@inner@#1}{%
    \tagsymbol{\infty@thm@thistag}%     
  }[section]
  
  % define the real environment
  % args: [pretty name] ref-label tag
  \newenvironment{#1}[3][]{%
    % set the current tag so amsthm notices it
    \def\infty@thm@thistag{#1:##2}%
    
    % define the new tag 
    \infty@tag@new{\infty@thm@thistag}{#1}{##3}%
   
    % begin the theorem
    \begin{infty@thm@inner@#1}[##1]%
    
    % and label it
    \label{\infty@thm@thistag}%

    % finally, insert a margin note
    \infty@tag@margin{\infty@thm@thistag}%
  }{%
    \end{infty@thm@inner@#1}%
    \undef\infty@thm@thistag%
  }
}

\theoremstyle{thm}
\infty@thm@new{thm}{Theorem}
\infty@thm@new{prop}{Proposition}
\infty@thm@new{lem}{Lemma}
\infty@thm@new{cor}{Corollary}
\infty@thm@new{conj}{Conjecture}

\theoremstyle{definition}
\infty@thm@new{defn}{Definition}
\infty@thm@new{exer}{Exercise}
\infty@thm@new{exmp}{Example}

\theoremstyle{remark}
\infty@thm@new{rem}{Remark}
\infty@thm@new{notation}{Notation}

\def\infty@macros@defined{}

\makeatother
