\makeatletter\@ifundefined{infty@macros@defined}{% imports
\usepackage{texor/texor}
\usepackage{hyperref}
\usepackage{marginnote}

\makeatletter

%%% PRELUDE %%%

\newcommand{\infty@abort}{
  \PackageError{18.90infty}
}

%%% TAGS %%%

% create a new tag
% args: tag name, tag type, tag number
\newcommand{\infty@tag@new}[3]{
  % check if this tag has been defined already,and error if it has
  \@ifundefined{infty@tag@canary@#1}{}{
    \infty@abort{tag '#3' defined already}{}
  }
  
  % check if this tag type is valid
  \@ifundefined{infty@tag@symbol@#2}{
    \infty@abort{tag type '#2' not found}{}
  }{}

  \expandafter\gdef\csname infty@tag@canary@#3\endcsname{#1}
  \expandafter\gdef\csname infty@tag@type@#1\endcsname{#2}
  \expandafter\gdef\csname infty@tag@num@#1\endcsname{#3}
}

% create a new tag type
% args: tag type name, tag type symbol
\newcommand{\infty@tag@newtype}[2]{
  \expandafter\gdef\csname infty@tag@symbol@#1\endcsname{#2}
}

\newcommand{\infty@tag@symoftype}[1]{\texor@call{infty@tag@symbol@#1}}

\newcommand{\infty@tag@type}[1]{\texor@call{infty@tag@type@#1}}
\newcommand{\infty@tag@num}[1]{\texor@call{infty@tag@num@#1}}
\newcommand{\infty@tag@sym}[1]{\infty@tag@symoftype{\infty@tag@type{#1}}}
\newcommand{\infty@tag@margin}[2][]{\marginnote{\tt \normalsize <\infty@tag@num{#2}>}}

\newcommand{\tagsymbol}[1]{%
  \infty@tag@labeled{\infty@tag@num{#1}}{\infty@tag@sym{#1}}%
}

% reference a tagged item, e.g.
% \tagref{thm:thom-isomorphism}
\newcommand{\tagref}[1]{%
  \@ifundefined{infty@tag@canary@#1}{%
    {\tt <ref:#1>}%
  }{%
    \hyperref[#1]{%
     \tagsymbol{#1} \ref*{#1}%
    }%
  }%
}

\newcommand{\infty@tag@labeled}[2]{%
  \def\infty@tag@label{{\tiny {\tt <#1>}}}%
  $\mathrlap{\expandafter\raisebox{-4pt}{\parbox[t][0pt]{\widthof{\infty@tag@label}}{\infty@tag@label}}}{\text{#2}}$%
}

%%% SECTIONS %%%

% wrap a section macro to include a tag
% args: name macro pretty-name
\newcommand{\infty@sec@wrap}[3]{
  % define a new tag type
  \infty@tag@newtype{#1}{#3}

  \expandafter\let\csname infty@sec@wrap@temp@#1\endcsname #2

  % args: label tag pretty-name
  \renewcommand{#2}[3]{%
    % create a new tag
    \infty@tag@new{#1:##1}{#1}{##2}%

    % call the wrapped macro
    \texor@call{infty@sec@wrap@temp@#1}{##3\infty@tag@margin{#1:##1}}

    % add the label
    \label{#1:##1}%
  }
}

\infty@sec@wrap{cha}{\chapter}{Chapter}
\infty@sec@wrap{sec}{\section}{Section}

%%% THEOREMS %%%

\newcommand{\infty@thm@new}[2]{
  % clear anything set by amsthm or texor
  \expandafter\undef\csname #1\endcsname
  \expandafter\undef\csname end#1\endcsname
  \expandafter\undef\csname c@#1\endcsname  

  % definie a new tag type
  \infty@tag@newtype{#1}{#2}

  % invoke the inner amsthm bit
  \newtheorem{infty@thm@inner@#1}{%
    \tagsymbol{\infty@thm@thistag}%     
  }[section]
  
  % define the real environment
  % args: [pretty name] ref-label tag
  \newenvironment{#1}[3][]{%
    % set the current tag so amsthm notices it
    \def\infty@thm@thistag{#1:##2}%
    
    % define the new tag 
    \infty@tag@new{\infty@thm@thistag}{#1}{##3}%
   
    % begin the theorem
    \begin{infty@thm@inner@#1}[##1]%
    
    % and label it
    \label{\infty@thm@thistag}%

    % finally, insert a margin note
    \infty@tag@margin{\infty@thm@thistag}%
  }{%
    \end{infty@thm@inner@#1}%
    \undef\infty@thm@thistag%
  }
}

\theoremstyle{thm}
\infty@thm@new{thm}{Theorem}
\infty@thm@new{prop}{Proposition}
\infty@thm@new{lem}{Lemma}
\infty@thm@new{cor}{Corollary}
\infty@thm@new{conj}{Conjecture}

\theoremstyle{definition}
\infty@thm@new{defn}{Definition}
\infty@thm@new{exer}{Exercise}
\infty@thm@new{exmp}{Example}

\theoremstyle{remark}
\infty@thm@new{rem}{Remark}
\infty@thm@new{notation}{Notation}

\def\infty@macros@defined{}

\makeatother
\def\infty@lock{}\begin{document}}{}\makeatother

\chapter{k-theory}{bcfe1b}{$K$-Theory}

\section{vec-bun-review}{7088ed}{Vector Bundles}

Before we can begin to tackle $K$-Theory, we will recall some basic facts
of vector bundles, though we will probably gloss over most of the details.
Throughout, we fix some nice field such as $\R$ or $\C$ (though, in general,
we assume to be working over $\C$ unless otherwise noted) and henceforth
supress it from the notation.

\begin{defn}{vec-buns}{d755ef}
  A \emph{vector bundle} $\xi$ consists of the following data:
  \begin{itemize}
    \item A surjection of spaces $p: E \srj B$. We call $p$, $E$, and $B$ the
          projection, total space, and base space, respectively.
    \item For each $b\in B$, a finite-dimentisonal vector space structure 
          on each fiber $p^{-1}\Par{b}$, which make each fiber into a 
          topological vector space.
  \end{itemize}  
  Furthermore, the map $p$ is \emph{locally trivial}, that is, for each 
  $b\in B$ there exists a neighborhood $U$ around it such that $p^{-1}\Par{U}$
  is homeomorphic to $B \times V$ for some vector space $V$.
\end{defn}

Since the dimension of each fiber varies continuously along $B$, it is clear that when $B$
is connected, all of the fibers have the same dimension. We call this number the \emph{rank}
of the bundle. We may refer to a rank $1$ bundle as a \emph{line bundle}, and to a rank $n$
bundle as an \emph{$n$-plane bundle}. 

The trivialization requirement suggests a simple class of vector bundles: those which
are \emph{globally trivial}:
\begin{exmp}{trivial-vec-bun}{47b41a}
  Given a space $X$ and a vector space $V$, we may form the \emph{trivial bundle}
  \[ \pr[1]: X \times V \to X \]
  whose projection being the product projection (hence the name!) and whose fiber is $V$. 
\end{exmp}

\begin{exmp}[The Tautological Bundle]{tautological-line-bundle}{6bb21c}
  Let $\Rp[n]$ be real projective space, the space of lines through the origin of
  $\R[n+1]$. Let $\tau = E \rmap{p} \Rp[n]$ be a line bundle where
  \[ E = \mkset{ \Par{\ell, v} \in \Rp[n] \times \R[n+1] }{ v \in \ell }, \] 
  where $p$ is given by the projection. This is called the \emph{tautological bundle} of
  $\Rp[n]$. The same construction may be done with $\Cp[n]$, too.
\end{exmp}

\begin{exmp}[The Tangent Bundle]{tangent-bundle}{ebdea1}
  Given a smooth manifold $M^n$, we may form its tangent bundle $TM$, made up of the union of
  its tangent spaces $T_x M$ of vectors tangent to $x \in M$\footnote{A rigorous treatment of
  the tangent bundle is outside the scope of this book. See \citeme for further reading.}
  The rank of $TM$ is equal to the dimension of the manifold $M$.
\end{exmp}

As objects over $B$, vector bundles have the maps you'd expect:

\begin{defn}{vec-bun-maps}{e65fec}
  A map of bundles over $B$ $\ph: \xi \to \xi'$ consists of a ladder
  \[\begin{tikzcd}
    F \ar{d}{\ph_F} \ar{r}{} & E \ar{d}{\ph_E} \ar{r}{p} & B \\
    F' \ar{r}{} & E' \ar[']{ur}{p'}
  \end{tikzcd}\]
  where the map of fibers $\ph_F$ is also a linear map. If such a map
  is invertible, we say $\xi$ and $\xi'$ are isomorphic, and denote it
  in the usual way.
\end{defn}

Given another space $B'$, we can and a map from it, we can perform a
\emph{change of base}:
\begin{defn}{vec-bun-change-of-base}{21049c}
  Given a vector bundle $\xi = E \rmap{p} B$ and a map $f: B' \to B$, we may
  form their pullback:
  \[\begin{tikzcd}
    E \times[B] B' \ar{r}{} \ar{d}{} & \ar{d}{p} E \\
    B' \ar{r}{f} & B
  \end{tikzcd}\]
  The vector bundle given by the vertical arrow $E \times[B] B' \to B'$ is denoted
  by $f^*\xi$ and is called the \emph{pullback along $f$}.
\end{defn}

The following are easily verified:
\begin{prop}{simple-props-of-vec-bun-change-of-base}{3814b2}
  \hfill
  \begin{itemize}
    \item If $B' \inj B$ is an inclusion, then $f^*\xi$ is isomorphic
          to the bundle given by restricting $p$ to to $B'$.
    \item For a sequence of bases $B'' \rmap{g} B' \rmap{f} B'$, there is
          a natural isomorphism
          \[ g^* f^* \xi \isom \Par{fg}^* \xi. \] 
  \end{itemize}
\end{prop}
\begin{proof}
  Exercise. % TODO: link to exercise at the end of the section.
\end{proof}


\makeatletter\@ifundefined{infty@lock}{}{\end{document}}\makeatother
