\makeatletter\@ifundefined{infty@macros@defined}{% imports
\usepackage{texor/texor}
\usepackage{hyperref}
\usepackage{marginnote}

\makeatletter

%%% PRELUDE %%%

\newcommand{\infty@abort}{
  \PackageError{18.90infty}
}

%%% TAGS %%%

% create a new tag
% args: tag name, tag type, tag number
\newcommand{\infty@tag@new}[3]{
  % check if this tag has been defined already,and error if it has
  \@ifundefined{infty@tag@canary@#1}{}{
    \infty@abort{tag '#3' defined already}{}
  }
  
  % check if this tag type is valid
  \@ifundefined{infty@tag@symbol@#2}{
    \infty@abort{tag type '#2' not found}{}
  }{}

  \expandafter\gdef\csname infty@tag@canary@#3\endcsname{#1}
  \expandafter\gdef\csname infty@tag@type@#1\endcsname{#2}
  \expandafter\gdef\csname infty@tag@num@#1\endcsname{#3}
}

% create a new tag type
% args: tag type name, tag type symbol
\newcommand{\infty@tag@newtype}[2]{
  \expandafter\gdef\csname infty@tag@symbol@#1\endcsname{#2}
}

\newcommand{\infty@tag@symoftype}[1]{\texor@call{infty@tag@symbol@#1}}

\newcommand{\infty@tag@type}[1]{\texor@call{infty@tag@type@#1}}
\newcommand{\infty@tag@num}[1]{\texor@call{infty@tag@num@#1}}
\newcommand{\infty@tag@sym}[1]{\infty@tag@symoftype{\infty@tag@type{#1}}}
\newcommand{\infty@tag@margin}[2][]{\marginnote{\tt \normalsize <\infty@tag@num{#2}>}}

\newcommand{\tagsymbol}[1]{%
  \infty@tag@labeled{\infty@tag@num{#1}}{\infty@tag@sym{#1}}%
}

% reference a tagged item, e.g.
% \tagref{thm:thom-isomorphism}
\newcommand{\tagref}[1]{%
  \@ifundefined{infty@tag@canary@#1}{%
    {\tt <ref:#1>}%
  }{%
    \hyperref[#1]{%
     \tagsymbol{#1} \ref*{#1}%
    }%
  }%
}

\newcommand{\infty@tag@labeled}[2]{%
  \def\infty@tag@label{{\tiny {\tt <#1>}}}%
  $\mathrlap{\expandafter\raisebox{-4pt}{\parbox[t][0pt]{\widthof{\infty@tag@label}}{\infty@tag@label}}}{\text{#2}}$%
}

%%% SECTIONS %%%

% wrap a section macro to include a tag
% args: name macro pretty-name
\newcommand{\infty@sec@wrap}[3]{
  % define a new tag type
  \infty@tag@newtype{#1}{#3}

  \expandafter\let\csname infty@sec@wrap@temp@#1\endcsname #2

  % args: label tag pretty-name
  \renewcommand{#2}[3]{%
    % create a new tag
    \infty@tag@new{#1:##1}{#1}{##2}%

    % call the wrapped macro
    \texor@call{infty@sec@wrap@temp@#1}{##3\infty@tag@margin{#1:##1}}

    % add the label
    \label{#1:##1}%
  }
}

\infty@sec@wrap{cha}{\chapter}{Chapter}
\infty@sec@wrap{sec}{\section}{Section}

%%% THEOREMS %%%

\newcommand{\infty@thm@new}[2]{
  % clear anything set by amsthm or texor
  \expandafter\undef\csname #1\endcsname
  \expandafter\undef\csname end#1\endcsname
  \expandafter\undef\csname c@#1\endcsname  

  % definie a new tag type
  \infty@tag@newtype{#1}{#2}

  % invoke the inner amsthm bit
  \newtheorem{infty@thm@inner@#1}{%
    \tagsymbol{\infty@thm@thistag}%     
  }[section]
  
  % define the real environment
  % args: [pretty name] ref-label tag
  \newenvironment{#1}[3][]{%
    % set the current tag so amsthm notices it
    \def\infty@thm@thistag{#1:##2}%
    
    % define the new tag 
    \infty@tag@new{\infty@thm@thistag}{#1}{##3}%
   
    % begin the theorem
    \begin{infty@thm@inner@#1}[##1]%
    
    % and label it
    \label{\infty@thm@thistag}%

    % finally, insert a margin note
    \infty@tag@margin{\infty@thm@thistag}%
  }{%
    \end{infty@thm@inner@#1}%
    \undef\infty@thm@thistag%
  }
}

\theoremstyle{thm}
\infty@thm@new{thm}{Theorem}
\infty@thm@new{prop}{Proposition}
\infty@thm@new{lem}{Lemma}
\infty@thm@new{cor}{Corollary}
\infty@thm@new{conj}{Conjecture}

\theoremstyle{definition}
\infty@thm@new{defn}{Definition}
\infty@thm@new{exer}{Exercise}
\infty@thm@new{exmp}{Example}

\theoremstyle{remark}
\infty@thm@new{rem}{Remark}
\infty@thm@new{notation}{Notation}

\def\infty@macros@defined{}

\makeatother
\def\infty@lock{}\begin{document}}{}\makeatother

\chapter{k-theory}{bcfe1b}{$K$-Theory}

The story of modern algebaric topology really begins with \emph{$K$-theory},
ostensibly a method for studying spaces using vector bundles instead of
test spaces (like singular cohomology does). As we will discuver, $K$-theory
is much more intimately related to cohomology than its construction might
suggest. It will be the first of many \emph{extraordinary} cohomology theories
we will study on our journey to the frontier of algebraic topology.

Even as a way to study spaces, $K$-theory is radically different from 
\emph{ordinary} cohomology. Not only does it have values in negative degree,
but it is also intimately contected to the topology of unitary matrices.

\section{vec-bun-review}{7088ed}{A Review of Vector Bundles}

Before we can begin to tackle $K$-theory, we will recall some basic facts
of vector bundles, though we will probably gloss over most of the details.
Throughout, we fix some nice field such as $\R$ or $\C$ (though, in general,
we assume to be working over $\C$ unless otherwise noted) and henceforth
supress it from the notation.

\begin{defn}{vec-buns}{d755ef}
  A \emph{vector bundle} $\xi$ consists of the following data:
  \begin{itemize}
    \item A surjection of spaces $p: E \srj B$. We call $p$, $E$, and $B$ the
          projection, total space, and base space, respectively.
    \item For each $b\in B$, a finite-dimentisonal vector space structure 
          on each fiber $p^{-1}\Par{b}$, which make each fiber into a 
          topological vector space.
  \end{itemize}  
  Furthermore, the map $p$ is \emph{locally trivial}, that is, for each 
  $b\in B$ there exists a neighborhood $U$ around it such that $p^{-1}\Par{U}$
  is homeomorphic to $B \times V$ for some vector space $V$.
\end{defn}

Since the dimension of each fiber varies continuously along $B$, it is clear that when $B$
is connected, all of the fibers have the same dimension. We call this number the \emph{rank}
of the bundle. We may refer to a rank $1$ bundle as a \emph{line bundle}, and to a rank $n$
bundle as an \emph{$n$-plane bundle}. 

The trivialization requirement suggests a simple class of vector bundles: those which
are \emph{globally trivial}:
\begin{exmp}{trivial-vec-bun}{47b41a}
  Given a space $X$ and a vector space $V$, we may form the \emph{trivial bundle}
  \[ \pr[1]: X \times V \to X \]
  whose projection being the product projection (hence the name!) and whose fiber is $V$. 
\end{exmp}

\begin{exmp}[The Tautological Bundle]{tautological-line-bundle}{6bb21c}
  Let $\Rp[n]$ be real projective space, the space of lines through the origin of
  $\R[n+1]$. Let $\tau = E \rmap{p} \Rp[n]$ be a line bundle where
  \[ E = \mkset{ \Par{\ell, v} \in \Rp[n] \times \R[n+1] }{ v \in \ell }, \] 
  where $p$ is given by the projection. This is called the \emph{tautological bundle} of
  $\Rp[n]$. The same construction may be done with $\Cp[n]$, too.
\end{exmp}

\begin{exmp}[The Tangent Bundle]{tangent-bundle}{ebdea1}
  Given a smooth manifold $M^n$, we may form its tangent bundle $TM$, made up of the union of
  its tangent spaces $T_x M$ of vectors tangent to $x \in M$\footnote{A rigorous treatment of
  the tangent bundle is outside the scope of this book. See \citeme for further reading.}
  The rank of $TM$ is equal to the dimension of the manifold $M$.
\end{exmp}

As objects over $B$, vector bundles have the maps you'd expect:

\begin{defn}{vec-bun-maps}{e65fec}
  A map of bundles over $B$, $\ph: \xi \to \xi'$, consists of a ladder
  \[\begin{tikzcd}
    F \ar{d}{\ph_F} \ar{r}{} & E \ar{d}{\ph_E} \ar{r}{p} & B \\
    F' \ar{r}{} & E' \ar[']{ur}{p'}
  \end{tikzcd}\]
  where the map of fibers $\ph_F$ is also a linear map. If such a map
  is invertible, we say $\xi$ and $\xi'$ are isomorphic, and denote it
  in the usual way. The category consisting of vector bundles and these maps
  will be denoted $\cVecBun[B]$. 
\end{defn}

Given another space $B'$, we can and a map from it, we can perform a
\emph{change of base}:
\begin{defn}{vec-bun-change-of-base}{21049c}
  Given a vector bundle $\xi = E \rmap{p} B$ and a map $f: B' \to B$, we may
  form their pullback:
  \[\begin{tikzcd}
    E \times[B] B' \ar{r}{} \ar{d}{} & \ar{d}{p} E \\
    B' \ar{r}{f} & B
  \end{tikzcd}\]
  The vector bundle given by the vertical arrow $E \times[B] B' \to B'$ is denoted
  by $f^*\xi$ and is called the \emph{pullback along $f$}.
\end{defn}

The following are easily verified:
\begin{prop}{simple-props-of-vec-bun-change-of-base}{3814b2}
  \hfill
  \begin{itemize}
    \item If $B' \inj B$ is an inclusion, then $f^*\xi$ is isomorphic
          to the bundle given by restricting $p$ to to $B'$.
    \item For a sequence of bases $B'' \rmap{g} B' \rmap{f} B'$, there is
          a natural isomorphism
          \[ g^* f^* \xi \isom \Par{fg}^* \xi. \] 
  \end{itemize}
\end{prop}
\begin{proof}
  Exercise. % TODO: link to exercise at the end of the section.
\end{proof}

The following theorem will give us access to a number of useful operations
on vector bundles:
\begin{thm}{vect-fnct-to-vec-bun-fnct}{38efab}
  Let $\cVect$ be the category of finite dimensional vector spaces, and
  let $T:\cVect \to \cVect$ be a functor which mapes the map
  \[ \Hom[V, W] \to \Hom[T\Par{V}, T\Par{W}] \]
  continuous (where $V$ and $W$ are regarded as products of the ground field).
  Then, this functor may be lifted to a functor $\cVecBun[B] \to \cVecBun[B]$.

  An analogous theorem also holds for $T$ with domain a product of copies of 
  $\cVect$ and its opposite.
\end{thm}
\begin{proof}
  Atiyah. \citeme
\end{proof}

From \tagref{thm:vect-fnct-to-vec-bun-fnct} we are entitled to to the direct sum 
(also known as the \emph{Whitney sum}), 
tensor product, hom-bundle, dual, and exterior power of
any vector bundle(s), and the attendant natural isomorphisms. 
%% TODO: something about the category of bundles over B? 

\section{complex-k-theory}{f407c6}{Complex $K$-Theory}

$K$-theory originates in algebraic geometetry, named after the German \emph{Klasse}, as in
isomorphism ``class'', by Grothendieck \citeme{}. $K$-theory originated as a tool to study
shaves on an algebaric variety; Michael Atiyah and Friedrich Hirzebruch transfered 
the construction over to topology, resulting in \emph{complex topological $K$-theory}.
The story of $K$-theory begins with two objects: the semiring of vector bundle classes, and
the Grothendieck group. Let us begin by fixing $\C$ as our ground field and contemplating our
favorite space $X$. 

We'd like to study the vector bundles over $X$ algebraically, but they do not form a group.
In fact, they don't even form a monoid; the Whitney sum is only associative and commutative up
to isomorphism. However, if we consider \emph{isomorphism classes} of vector bundules, we
obtain an algebraic structure we can get a grip on:
\begin{defn}{monoid-vec-bun-iso}{190b0f} 
  Let $\cVect[X] = \poi[0] \cVecBun[X]$ denote the set of isomorphism classes of 
  $\cVecBun[X]$.\footnote{This is sometimes called the \emph{core} of a category.}
  This set forms a semiring (i.e. an abelian monoid with a multiplication) under the
  Whitney sum and the tensor product, respectively.
\end{defn}

However, the theory of semirings does not fit into our existing machinery of
homological algebra: it is not an abelian group. The next step is to use a construction
of Grothendieck to ``groupify'' it.
\begin{defn}{grothendieck-group}{9d1b08}
  Let $M$ be an abeilan monoid. The \emph{Grothendieck group} $K\Par{M}$ is defined by
  the following universal property:
  \[\begin{tikzcd}
    & G \\
    K\Par{M} \ar[dashed]{ur}{} \ar{r}{} & M \ar{u}{}
  \end{tikzcd}\]
  where $G$ is any group and $M \to G$ is any map of monoids. In other words, $K\Par{M}$
  consists of formally adjoining the ``missing'' inverses to $M$: for each $m\in M$, 
  there is a $-m\in K\Par{M}$. We may call these ``negative'' classes \emph{virtual classes}.  

  Formally, $K\Par{M}$ is a quotient of $M \times M$ where $\Par{m, n}$ corresponds to $m - n$. 
  Alternatively, we may define $K\Par{M}$ as a quotient of the free group ${\Z}M$, where we set
  $m + n \sim m \oplus n$, here $+$ and $\oplus$ being the additions of ${\Z}M$ and $M$,
  respectively.
\end{defn}

\begin{defn}{zeroth-k-group}{6fca6d}
  The \emph{Grothendieck group of $X$}, or the \emph{zeroth $K$-theory of $X$} is the
  Grothendieck group of its vector bundles, denoted 
  \[ \Kthy[0][X] = K\Par{\cVect[X]}. \] 
  Moreover, the tensor product provides $\Kthy[0][X]$ with a multiplication, where the
  trivial line bundle $\ep = X \times \C$ is our one. This makes the Grothendieck group a ring.
  In fact, this is a contravariant functor. Given a map of spaces $f: Y \to X$, we may perform
  a change of base of all of the vector bundles to obtain a map $\Kthy[0][X] \to \Kthy[0][Y]$.
\end{defn}

\begin{exmp}{grothendieck-group-of-pt}{e0eaa1}
  Because every vector bundle over the point is trivial, its Grothendieck group is just the
  integers: $\Kthy[0][\pt] = \Z$.
\end{exmp}

While we have succeeded in making a ring out of vector bundles, this is merely a 
slice of $K$-theory\footnote{Spoilers: it's a really big slice.}! Let's work through some
properties of $\Kthy[0][X]$. Throughout, $\Squ{\xi}$ is the (non-virtual) class represented
by the bundle $\xi$.

\begin{prop}{elements-of-K-0}{aa6eaf}
  Every element of $\Kthy[0][X]$ is of the form $\Squ{\xi} - n$ for some bundle $\xi$ and
  some natural number $n$.
\end{prop}
\begin{proof}
  By construction, 
  every element of $\Kthy[0][X]$ is of the form $\Squ{\xi} - \Squ{\ze}$. Pick some 
  bundle $\eta$ such that $\ze \oplus \eta$ is trivial, i.e., it is isomorhic to $n\ep$. 
  Then,
  \[ \Squ{\xi} - \Squ{\ze} = \Squ{\xi} + \Squ{\eta} - \Squ{\ze} - \Squ{\eta} = 
      \Squ{\xi \oplus \eta} - \Squ{n\ep} = \Squ{\xi \oplus \eta} - n. \] 
\end{proof}

\begin{prop}{equality-in-K-0}{d64ddf}
  Two classes $\Squ{\xi}, \Squ{\ze}$ in $\Kthy[0][X]$ are equal if and only if
  $\xi \oplus n\ep \isom \ze \oplus n\ep$ for some natural number $n$.
\end{prop}
\begin{proof}
  If $\Squ{\xi} = \Squ{\ze}$, then $\xi \oplus \eta \isom \ze \oplus \eta$ for some bundle
  $eta$ (by construction). The rest of the argument follows as in \tagref{prop:elements-of-K-0}.
\end{proof}

\section{k-thy-periodicity-theorem}{359137}{A Periodicity Theorem}

The fundamental property of $K$-theory is \emph{Bott periodicity}. The only form we can
state right now is
\[ \Kthy[0][X \times {\Sph[2]}] \isom \Kthy[0][X] \otimes \Kthy[0][{\Sph[2]}], \]
but, as we will eventually learn, is a deep fact about the structure of $\sU$, the unitary
group.

We begin by considering the \emph{projective bundle} associated to a vector bundle.
\begin{defn}{projective-bundles}{e824c2}
  Let $\xi = E \to B$ be a vector bundle. We can remove a copy of $B$ from $E$ by deleting
  the zero element of each fiber. Then $\units{\C}$ acts on $E\sans B$ by fiberwise scaling.
  define $\Pr[][][E]$ as the quotient of $E\sans B$ by this action, and moreover, let
  $\Pr[][][\xi]$ be the bundle $\Pr[][][E] \to B$. This is the \emph{projectivization} of
  $xi$. Note that this is merely a fibre bundle; the fibers are not vector spaces, but
  projective spaces. 
\end{defn}

\begin{prop}{projectivizaton-up-to-line-bundle}{0e40ac}
  For any vector bundle $\xi$ and line bundule $\ell$,
  \[ \Pr[][][\xi \otimes \ell] \isom \Pr[][][\xi]. \]
\end{prop}
\begin{proof}
  We apply \tagref{thm:vect-fnct-to-vec-bun-fnct} to reduce this to a question about 
  projectivizations of vector spaces.
  Let $\Pr[][][V]$ be the projectivization of the vector space $V$. Given a nonzero $w\in W$,
  the assignment $v\mapsto v\otimes w$ gives a map $V \to V \otimes W$. If $\dim W = 1$,
  this map is an (unnatural) isomorphism. If we projectivize, however, we no longer care
  about the choice of sign of $w$, so $\Pr[][][V] \isom \Pr[][][V \otimes W]$.
\end{proof}

We will not prove the following theorem until much later, but we state it now
and ponder its corollaries.
\begin{thm}{main-thm-bott-periodicity}{cadc0b}
  Let $\ell$ be a line bundle over $X$, and consider the projective bundle 
  $\Pr[][][\ell \oplus \ep]$. Its Grothendieck group is completely determined as a 
  $\Kthy[0][X]$-algebra, given as the following quotient of a polynomial
  algebra:\footnote{Note that the generator of this algebra is $\Squ{\eta}$; it is
  \emph{not} a formal power series ring, despite the double square brackets.} 
  \[ \Kthy[0][{\Pr[][][\ell\oplus\ep]}] = 
      \frac{\Kthy[0][X]\Squ{\Squ{\eta}}}
        {\Par{\Par{\Squ{\eta}-1}\Par{\Squ{\eta}\Squ{\ell} - 1}}}. \]
  Here, $\Squ{\eta}$ is another line bundle, which we will contruct later.
\end{thm}

This has two important corollaries. Consider the vector bundle $\ep \oplus \ep$ over
$X$, i.e. $X \times \C[2] \to X$. If we projectivize, we wind up with the projective
bundle $X \times \Cp[1]$. Recalling that the complex projective line is $\Sph[2]$, we
have the following corrolaries:

\begin{cor}{grothendieck-group-of-S2}{dbc762}
  Setting $X = \pt$ and $\ell = \ep$ above, we find that
  \[ \Kthy[0][{\Sph[2]}] = \frac{{\Z}\Squ{eta}}{\Par{\Par{\Squ{eta} - 1}^2}}. \]
\end{cor}

\begin{cor}{grothendieck-group-kunneth}{4f605f}
  The following K\"unneth theorem holds, where the isomorphism is
  the obvious multiplication map.
  \[ \Kthy[0][X] \otimes \Kthy[0][{\Sph[2]}] \isom
      \Kthy[0][{X \times \Sph[2]}]. \]
\end{cor}
 
%% TODO: prove big main theorem (gross)

\section{k-thy-is-cohomology}{1d1dc4}{An Extraordinary Theory}

We will now develop $K$-theory proper, using the periodicity theorems we've built up.
We'll also show how $K$-theory is the next step towards modern algebaric topology, as
the first example of an \emph{extraordinary cohomology theory}.

Let $\cTop$ be our category of spaces, as usual, and let $\cTop_*$ be the pointed variant.
Finally, let $\cTop^2$ be the category of pairs of spaces. Finally, given a space $X$, let
$X^+$ be that space with a disjoint basepoint.

The following construction is entirely analogous to the equivalent one for singular
cohomology.
\begin{defn}{reduced-grothendieck-group}{e99f3a}
  Given a pointed space $X$ with basepoint $x$, let the \emph{reduced Grothendieck group}
  be given as
  \[ \rKthy[0][X] = \ker[{\Kthy[0][X] \to \Kthy[0][x]}], \]
  i.e., as the kernel of the image of the inclusion of the basepoint. By definition, the
  collapse map $X \to x$ induces a splitting
  \[ \Kthy[0][X] = \rKthy[0][X] \oplus \rKthy[0][x]. \]
  We define the \emph{relative Grothendieck group} of a pair as
  \[ \Kthy[0][X, Y] = \rKthy[0][X/Y]. \]
\end{defn}

The usual facts, e.g. $\Kthy[0][X] = \rKthy[0][X^+]$, $\Kthy[0][X, \eset] = \Kthy[0][X]$,
are obvious. We can now define $K$-theory in \emph{negative degree} using the reduced
Grothendieck Group.
\begin{defn}{negative-k-thy}{55a1bf}
  Let $X$ be a pointed space. Then, define \emph{reduced $K$-theory}
  \[ \rKthy[-n][X] = \rKthy[0][{\Susp[n]X}] \]
  for $n > 0$. $K$-theory itself and its relative version are defined in
  terms of this:
  \[ \Kthy[-n][X] = \rKthy[-n][X^+], \quad
       \Kthy[-n][X, Y] = \rKthy[-n][X/Y]. \]
  These are all contravariant functors with the expected effect on maps, and
  are all algebras over the relevant Grothendieck group, which takes the role of the
  zeroth $K$-theory.
\end{defn}

\makeatletter\@ifundefined{infty@lock}{}{\end{document}}\makeatother
